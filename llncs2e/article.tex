%\documentclass[10pt,a4paper]{article}
%\usepackage[utf8]{inputenc}
%\usepackage[english]{babel}
%\usepackage{amsmath}
%\usepackage{amsfonts}
%\usepackage{amssymb}
%\usepackage{graphicx}
%\author{Frederik Hagelskjær}
\documentclass{llncs} 
\usepackage{llncsdoc}

\begin{document}

\title{Title}
\author{Frederik Hagelskjær}
\institute{University of Southern Denmark, Campusvej 55, 5230 Odense M, Denmark \\frhag10@student.sdu.dk}
\maketitle

\begin{abstract}

This article is the course AI2 at the University of Southern Denmark. 

\end{abstract}

\section*{Introduction}

The LUDO game. 

Using AI to develop a Ludo playing agent. 

There are no real requirements to the agents performance, but the results of the developed AI should be extensively documented. 

The goal of AI is an introduction to artificial neural networks, and therefore the agents developed in this article will use such an approach.

The complexity of the LUDO game, makes it impossible to make a analytical solution by simple determining every solution for every state, thus making decision making a lookup table, as shown in \citep{6031999}. This article tries to determine an appropriate neural network for an agent playing on level with other solutions.

Originally the neural networks were trained using different error minimization procedures.

Training the neural network.

Back-propagating was used. 

Trying to emulate a human playing ludo.



\end{document}

\bibliography{article_bibtex}
\bibliographystyle{plain}